%%%%%%%%%%%%%%%%%%%%%%%%%%%%%%%%%%%%%%%%%
% Journal Article
% ICT Tools
% Assignment 3: Comparative analysis of video creation tools
%
% Gahan M. Saraiya
% 18MCEC10
%
%%%%%%%%%%%%%%%%%%%%%%%%%%%%%%%%%%%%%%%%%
%----------------------------------------------------------------------------------------
%       PACKAGES AND OTHER DOCUMENT CONFIGURATIONS
%----------------------------------------------------------------------------------------
\documentclass[paper=letter, fontsize=12pt]{article}
\usepackage[english]{babel} % English language/hyphenation
\usepackage{amsmath,amsfonts,amsthm} % Math packages
\usepackage[utf8]{inputenc}
\usepackage{float}
\usepackage{lipsum} % Package to generate dummy text throughout this template
\usepackage{blindtext}
\usepackage{graphicx} 
\usepackage{caption}
\usepackage{subcaption}
\usepackage[sc]{mathpazo} % Use the Palatino font
\usepackage[T1]{fontenc} % Use 8-bit encoding that has 256 glyphs
\usepackage{bbding}  % to use custom itemize font
\linespread{1.05} % Line spacing - Palatino needs more space between lines
\usepackage{microtype} % Slightly tweak font spacing for aesthetics
\usepackage[hmarginratio=1:1,top=32mm,columnsep=20pt]{geometry} % Document margins
\usepackage{multicol} % Used for the two-column layout of the document
%\usepackage[hang, small,labelfont=bf,up,textfont=it,up]{caption} % Custom captions under/above floats in tables or figures
\usepackage{booktabs} % Horizontal rules in tables
\usepackage{float} % Required for tables and figures in the multi-column environment - they need to be placed in specific locations with the [H] (e.g. \begin{table}[H])
\usepackage{hyperref} % For hyperlinks in the PDF
\usepackage{lettrine} % The lettrine is the first enlarged letter at the beginning of the text
\usepackage{paralist} % Used for the compactitem environment which makes bullet points with less space between them
\usepackage{abstract} % Allows abstract customization
\renewcommand{\abstractnamefont}{\normalfont\bfseries} % Set the "Abstract" text to bold
\renewcommand{\abstracttextfont}{\normalfont\small\itshape} % Set the abstract itself to small italic text
\usepackage{titlesec} % Allows customization of titles

\renewcommand\thesection{\Roman{section}} % Roman numerals for the sections
\renewcommand\thesubsection{\Roman{subsection}} % Roman numerals for subsections

\titleformat{\section}[block]{\large\scshape\centering}{\thesection.}{1em}{} % Change the look of the section titles
\titleformat{\subsection}[block]{\large}{\thesubsection.}{1em}{} % Change the look of the section titles
\newcommand{\horrule}[1]{\rule{\linewidth}{#1}} % Create horizontal rule command with 1 argument of height
\usepackage{fancyhdr} % Headers and footers
\pagestyle{fancy} % All pages have headers and footers
\fancyhead{} % Blank out the default header
\fancyfoot{} % Blank out the default footer

\fancyhead[C]{Institute of Technology, Nirma University $\bullet$ October 2018} % Custom header text

\fancyfoot[RO,LE]{\thepage} % Custom footer text
%----------------------------------------------------------------------------------------
%       TITLE SECTION
%----------------------------------------------------------------------------------------
\title{\vspace{-15mm}\fontsize{16pt}{6pt}\selectfont Assignment 3: Comparative analysis of video creation tools} % Article title
\author{
	\large
	{\textsc{Gahan Saraiya, 18MCEC10 }}\\[2mm]
	%\thanks{A thank you or further information}\\ % Your name
	\normalsize \href{mailto:18mcec10@nirmauni.ac.in}{18mcec10@nirmauni.ac.in}\\[2mm] % Your email address
}
\date{}
\hypersetup{
	colorlinks=true,
	linkcolor=blue,
	filecolor=magenta,      
	urlcolor=cyan,
	pdfauthor={Gahan Saraiya},
	pdfcreator={Gahan Saraiya},
	pdfproducer={Gahan Saraiya},
}
%----------------------------------------------------------------------------------------
\usepackage[utf8]{inputenc}
\usepackage[english]{babel}
\usepackage[utf8]{inputenc}
\usepackage{fourier} 
\usepackage{array}
\usepackage{makecell}

\renewcommand\theadalign{bc}
\renewcommand\theadfont{\bfseries}
\renewcommand\theadgape{\Gape[4pt]}
\renewcommand\cellgape{\Gape[4pt]}
\newcommand*\tick{\item[\Checkmark]}
\newcommand*\arrow{\item[$\Rightarrow$]}
\newcommand*\fail{\item[\XSolidBrush]}
\newcommand*{\screencastify}{\href{https://www.screencastify.com}{Screencastify}}
\newcommand*{\hyfy}{\href{https://www.hyfy.io}{HYFY}}
\newcommand*{\kazam}{\href{https://launchpad.net/kazam}{Kazam Screencaster}}

\usepackage{minted} % for highlighting code sytax

\begin{document}
	\maketitle % Insert title
	\thispagestyle{fancy} % All pages have headers and footers
	
	
	\section{Introduction}
	\paragraph{} Goal of this assignment is to compare various video creation tool.
	Tools which are going to be covered are listed below:
	\begin{itemize}
		\arrow \screencastify
		\arrow \hyfy
		\arrow \kazam
	\end{itemize}
	
	\section{Comparison}
	
	\begin{table}[H]
		\centering
		\bgroup
		\setlength{\parindent}{-5em} 
		\caption*{Comparitive analysis}
		\begin{tabular}{r | c | c | c }
			& \textbf{\screencastify} & \textbf{\hyfy} & \textbf{\kazam} \\
			\hline
			\hline
			Features & 
				\makecell[l]{
%					Desktop, browser tab or webcam capture
					Record Browser Tab
					\\ Record Desktop screen
					\\ Record webcam
					\\ Narrate with your mic's audio
					\\ Customize resolution and FPS
					\\ Embed webcam into screencast
					\\ Annotate screen with a pen tool
					\\ Focus a spotlight on your mouse
					\\ Save/Share to Google Drive
					\\ Save/Share to Google Classroom
					\\ Share to YouTube
				} & 
				\makecell[l]{
					\\ Record any browser window
					\\ Record full screen images
					\\ Record voice
					\\ No video files to convert/upload
					\\ Share from wherever you work
					\\ no software to install to view
					\\ Share directly to JIRA
					\\ Share directly to HipChat
					\\ Share directly to Slack
					\\ Share bite-sized GIFs
				} & 
				\makecell[l]{
					\\ Simple and compact user interface
					\\ Open Source
					\\ add notes or comments
					\\ Quality with compressed size
					\\ No recording Limit
					\\ Record voice
					\\ Record a single window 
					\\ Record selected screen area
				}
			\\
			\hline
			Limitations &
				\makecell[l]{
					10 minute video length limit
					\\ 50 videos / month
					\\ Screencastify watermark
				} & 
				\makecell[l]{
					10-minute recording time/video
					\\ video viewing limited to 7 days
				} & 
				\makecell[l]{
					Lack of WebCam recording 
				}
			\\
			\hline
			\makecell{Premium \\Features} &
			\makecell[l]{
				Unlimited recording length
				\\ Unlimited number of videos
				\\ Unlimited video editing
				\\ No watermark
				\\ Export as .MP4 / animated GIF
				\\ Merge multiple videos
				\\ Cut/split clips
				\\ Trim and crop video
			} & 
			\makecell[l]{
				Unlimited video viewing
				\\ Select and limit viewers' access
				\\ Add brand logo
				\\ colors to customers’ views
				\\ 60-minute recording time/video
				\\ Animated GIF creation
				\\ Cropping
			} & 
			\makecell[l]{
				free to use for GNU/Linux
			}
		\\
		\hline
		\makecell{Platform \\Support} &
		\makecell[l]{
			Any platform with Chrome
		} & 
		\makecell[l]{
			Any platform with Chrome
		} & 
		\makecell[l]{
			GNU/Linux
		}
		\end{tabular}
	\egroup
	\end{table}
	

\end{document}
